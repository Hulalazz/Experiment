% !TEX TS-program = pdflatex
% !TEX encoding = UTF-8 Unicode

% This is a simple template for a LaTeX document using the "article" class.
% See "book", "report", "letter" for other types of document.

\documentclass[11pt]{article} % use larger type; default would be 10pt

\usepackage[utf8]{inputenc} % set input encoding (not needed with XeLaTeX)

%%% Examples of Article customizations
% These packages are optional, depending whether you want the features they provide.
% See the LaTeX Companion or other references for full information.

%%% PAGE DIMENSIONS
\usepackage{geometry} % to change the page dimensions
\geometry{a4paper} % or letterpaper (US) or a5paper or....
% \geometry{margin=2in} % for example, change the margins to 2 inches all round
% \geometry{landscape} % set up the page for landscape
%   read geometry.pdf for detailed page layout information

\usepackage{graphicx} % support the \includegraphics command and options

% \usepackage[parfill]{parskip} % Activate to begin paragraphs with an empty line rather than an indent

%%% PACKAGES
\usepackage{booktabs} % for much better looking tables
\usepackage{array} % for better arrays (eg matrices) in maths
\usepackage{paralist} % very flexible & customisable lists (eg. enumerate/itemize, etc.)
\usepackage{verbatim} % adds environment for commenting out blocks of text & for better verbatim
\usepackage{subfig} % make it possible to include more than one captioned figure/table in a single float
% These packages are all incorporated in the memoir class to one degree or another...

% Write algorithm
\usepackage{algorithm}
%format of the algorithm
\usepackage{algorithmic}
%format of the algorithm
\usepackage{multirow}
%multirow for format of table
\usepackage{amsmath}
\usepackage{xcolor}

\DeclareMathOperator*{\argmin}{argmin}
\renewcommand{\algorithmicrequire}{\textbf{Initialize:}}
%/Use Input in the format of Algorithm
\renewcommand{\algorithmicensure}{\textbf{Output:}}
%\usepackage{algpseudocode}

%%% HEADERS & FOOTERS
\usepackage{fancyhdr} % This should be set AFTER setting up the page geometry
\pagestyle{fancy} % options: empty , plain , fancy
\renewcommand{\headrulewidth}{0pt} % customise the layout...
\lhead{}\chead{}\rhead{}
\lfoot{}\cfoot{\thepage}\rfoot{}

%%% SECTION TITLE APPEARANCE
\usepackage{sectsty}
\allsectionsfont{\sffamily\mdseries\upshape} % (See the fntguide.pdf for font help)
% (This matches ConTeXt defaults)

%%% ToC (table of contents) APPEARANCE
\usepackage[nottoc,notlof,notlot]{tocbibind} % Put the bibliography in the ToC
\usepackage[titles,subfigure]{tocloft} % Alter the style of the Table of Contents
\renewcommand{\cftsecfont}{\rmfamily\mdseries\upshape}
\renewcommand{\cftsecpagefont}{\rmfamily\mdseries\upshape} % No bold!



%%% END Article customizations

%%% The "real" document content comes below...

\title{Gradient-based Optimization Methods in Deep Learning}
\author{Zhang Jinxiong}
%\date{} % Activate to display a given date or no date (if empty),
         % otherwise the current date is printed

\begin{document}
\maketitle

\abstract
Stochastic gradient descent is to optimize the finite sum objective functions based on gradient descent.
Stochastic gradient descent (SGD) is the popular optimization algorithm in deep learning.
It is easy to implement and of much advantages in differentiable optimization within high dimension space.
\section{Introduction}

Gradient descent or steepest descent is the fundamental optimization only using the gradient of the object function.
 It is simple and especially efficient in convex optimization problem.
%But when the dimension of the objective function is too high, it is not easy to compute its gradient.
However, there are many cases that it is difficult or expensive to compute the gradient in high dimension space.
To void that, we just grasp some ``local" or ``partial" information instead of the full information in high dimension space.

For example, the coordinate descent algorithm is to optimize the multi-variable function in one specific variable or coordinate while the rest are fixed sequentially.

It is natural for us to take the advantage of ``partial" gradient information in place of the full gradient information with affordable computation cost.
This paper is aimed to review some stochastic gradient descent methods.


\section{Determinant Gradient Method}
The stochastic gradient descent is rooted in classical or determinant method.

\subsection{Gradient Descent}
Gradient descent is based on the local information -  it is  the negative gradient that is most speedy to decrease for a function in a samll region around a given point.
Let $f(x)$ be the objective function and $\nabla f(x) = g(x)$.
\begin{algorithm}[htb]
\caption{Gradient Descent}
%算法的标题
\label{GD}
%给算法一个标签,这样方便在文中对算法的引用
\begin{enumerate}
\item Give the initial value: $x_0$;
\item Update the parameters with small step: $$x_{k+1}=x_{k} - \alpha g(x_{k}).$$
\end{enumerate}

\end{algorithm}

Note: $\alpha$ must be small enough to ensure that it is descent: $f(x_{k+1})< f(x_{k})$.

The step is key factor in the convergence proof of the gradient descent.
One common setting is diminishing but not summable: $$\sum_{k=0}^{\infty} {\alpha}_{k}= \infty,\sum_{k=0}^{\infty} {\alpha}_{k}^{2}< \infty. $$

\subsection{The Momentum Methods}
Momentum is form the physics.
It is used to accelerate the speed of convergence.
\begin{algorithm}[htbp]
\caption{Gradient Descent with momentum}
\label{momentum}
\begin{enumerate}
\item Give the initial value: $x_0$;
\item Compute the momentum: $v_k= \gamma  v_{k-1} -\alpha g({\theta}_k)$;
\item Update the parameters with small step: $${\theta}_{k+1}={\theta}_k + v_k .$$
\end{enumerate}

\end{algorithm}

Nesterov's momentum is different from the classical momentum in the position where it is computed.
$$v_k= \gamma  v_{k-1}-\alpha g({\theta}_k+v_{k-1})$$
$${\theta}_{k+1}={\theta}_k + v_k$$


\subsection{The Dynamics in Optimization}
It is simple to prove the convergence rate of the above methods in a unified framework.
%CONtinuous and Discrete Dynamics for online Learning and Convex Optimization

\subsection{Some Examples}
Some examples will show the details of the methods.
And the step length and profitable direction are the focus of the gradient-based optimization methods.

\subsubsection{The Projected Gradient Methods}
This example is from Bingsheng He. %http://maths.nju.edu.cn/~hebma/slides/08C.pdf
It shows that the steps  of the steepest descent method affect the speed of convergence dramatically.

\subsubsection{Newton's Method for Quadratic Optimization}
It takes one step in the direction of Newton's method for quadratic optimization.
In theory, Newton's method just take one step to get the optimum.However, the gradient descent may take more steps.

\section{Stochastic Gradient Descent}

Stochastic methods are not to compute the exact gradient but to estimate the gradient by sampling.
It is always applied to the optimization in the finite sum form. For example, it is to solve the mean square error of least square method in large scale.

The object function is called empirical risk function with the
form $f(x_i;\theta)=\sum_{i=1}^{n}f_{i}(\theta)$, where $x_i$ is constant parameters given by training sample and $f_{i}$s have the same independent variables $\theta$.

Let $\nabla f_{i}(x) = g_{i}(x)$. The stochastic gradient descent with constant step is shown below.
\begin{algorithm}[htb]
\caption{Primary Stochastic Gradient Descent}
%算法的标题
\label{SGD}
\begin{enumerate}
  \item Give the initial value: $x_0$;
  \item Randomly choose term index $k_{i}$ in the $k$th iteration from ${1,2,\cdots, n}$;
  \item Update the parameters: $${\theta}_{k+1}={\theta}_{k} - \alpha g_{k_{i}}({\theta}_{k}).$$
\end{enumerate}
\end{algorithm}

Note the difference with gradient descent: the index $k_{i}$ is a random variable!
We can randomly choose a minibatch of training set to compute the gradient instead of the single sample:
${\theta}_{k+1}={\theta}_{k} - \alpha \frac{1}{m}\sum_{i=1}^{m}g_{k_{i}}({\theta}_{k})$.

%[See also at Wikipedia.](https://www.wikiwand.com/en/Stochastic_gradient_descent) and [SGD in CS231](https://cs231n.github.io/neural-networks-3/#sgd).

\subsection{The Difference with Full Gradient Descent}
The sufficient difference in SGD and GD is estimated gradient in SGD is far from the full original gradient in GD.
The noise of the estimated gradient makes the convergence more slow.
%Large-Scale Machine Learning with Stochastic Gradient Descent  by L\'{e}on Bottou
Leon Bottou have claimed that the  convergence speed of stochastic gradient descent is in fact limited by the noisy approximation of the true gradient.
%Optimization for Large-Scale Machine Learning by Bottou, Frank E. Curtis and Jorge Nocedal.

\section{Adaptive Learning Rate}

There are some strategies of choosing the step sizes.

\subsubsection{Adagrad}
It is based on the experience that infrequent parameters can give more information than the frequent parameters.
Thus it is designed to perform larger updates for infrequent and smaller updates for frequent parameters.

It individually adapts the learning rates of all model parameters by scaling them inversely proportional to the square root of the sum of all the historical squared values of the gradient.
\begin{algorithm}[htb]
\caption{Adagrad Algorithm}
%算法的标题
\label{Adagrad}
%给算法一个标签,这样方便在文中对算法的引用

  \begin{enumerate}
  \item Sample a minibatch of m examples from the training set $\{x_{k_1}, x_{k_2}, \dots, x_{k_m}\}$ with corresponding targets $y_{k_i}$;
  \item Compute the gradient:$g_{k} = \frac{1}{m}\sum_{i=1}^{m}g_{k_{i}}({\theta})$;
  \item Accumulate squared gradient: $r_{k+1}= r_k+ g_{k+1}\odot g_{k+1}$, where $\odot$ is multiplication applied element-wise;
  \item Update: ${\theta}_{k+1}={\theta}_k - \frac{\varepsilon}{\delta + \sqrt{r_{k+1}}}\odot g_k$ (Division and square root applied element-wise).
\end{enumerate}
\end{algorithm}

\subsubsection{RMSprop}

RMSProp uses an \textbf{exponentially decaying average}to discard history from the extreme past.
\begin{algorithm}[htb]
\caption{RMSprop Algorithm}
%算法的标题
\label{RMSprop}
%给算法一个标签,这样方便在文中对算法的引用
  \begin{enumerate}
  \item Sample a minibatch of m examples from the training set $\{x_{k_1}, x_{k_2}, \dots, x_{k_m}\}$ with corresponding targets $y_{k_i}$;
  \item Compute the gradient:$g_{k} = \frac{1}{m}\sum_{i=1}^{m}g_{k_{i}}({\theta})$;
  \item Accumulate squared gradient: $r_{k+1}= \rho r_k+(1-\rho)g_{k+1}\odot g_{k+1}$, where $\odot$ is multiplication applied element-wise;
  \item Update: ${\theta}_{k+1}={\theta}_k - \frac{\varepsilon}{\delta + \sqrt{r_{k+1}}}\odot g_k$ (Division and square root applied element-wise).
\end{enumerate}

\end{algorithm}

\subsubsection{Adam}
Adaptive Moment Estimation (Adam) is another method that computes adaptive learning rates for each parameter.
It can be seen as a variant on the combination of RMSProp and momentum.

\begin{algorithm}
\caption{Adam}
\label{Adam}
 Sample a minibatch of m examples from the training set $\{x_{k_1}, x_{k_2}, \dots, x_{k_m}\}$ with corresponding targets $y_{k_i}$;

 Compute the gradient:$g_{k} = \frac{1}{m}\sum_{i=1}^{m}g_{k_{i}}({\theta})$;

 Update biased first moment estimate: $s_{k+1}={\rho}_1 s_k + (1-{\rho}_1)g_k$;

 Update biased second moment estimate: $r_{k+1}={\rho}_2 r_k + (1-{\rho}_2)g_k\odot g_k$;

 Correct bias in first moment: $\hat s_{k+1}=\frac{s_{k+1}}{1-{\rho}_{1}^k}$;

 Correct bias in second moment: $\hat r_{k+1}=\frac{r_{k+1}}{1-{\rho}_{2}^k}$;

 Update: ${\theta}_{k+1}={\theta}_k -\varepsilon \frac{\hat s_{k+1}}{\sqrt{\hat r_{k+1}}+\delta}$ .
%\end{algorithmic}
\end{algorithm}

Adam is most popular in deep learning community.%http://ruder.io/deep-learning-optimization-2017/index.html#fn:7




\section{Variance Reduction}
The stochastic gradient descent is not always descent due to its inherent variance.

\subsubsection{Stochastic Average Gradient}
Like stochastic gradient (SG) methods, the SAG method’s iteration cost is independent of the number of terms in the sum.
However, by incorporating a memory of previous gradient values the SAG method achieves a faster convergence rate than black-box SG methods.


The SAG iterations take the form:
$${\theta}_{k+1}={\theta}_k - \frac{{\alpha}_k }{n}\sum _{i=1}^{n}y_{i_k} ,$$
where at each iteration a random index $i_k$ is selected and we set:
$$y_{i_k}=
\begin{cases}
g_{i}({\theta}_k),  & \text{if $i=i_k$}; \\
y_{i_{k-1}}, & \text{otherwise}.
\end{cases}$$

\subsection{SVRG}%\href{}{http://riejohnson.com/rie/stograd_nips.pdf}
Rie Johnson and Tong Zhang introduced an explicit variance reduction method for stochastic gradient descent.
It is proved that this method enjoys the same fast convergence rate as those of stochastic dual coordinate ascent (SDCA) and Stochastic Average Gradient (SAG) for \emph{smooth and strongly convex} functions.

Its aim is to reduce the variance of the gradient so that the process is more stable.
%\begin{algorithm}[t]
%\caption{SVRG}
%\label{SVRG}
%\textbf{Input:} $\hat{\theta}={\hat{\theta}}_{s-1}$, $\hat{\mu}=\frac{1}{n}\sum_{i=1}^{m} g_i (\hat{\theta})$,${\theta}_0=\hat{\theta}$;
%
%  For $k\in \{ 1, 2, \cdots, m\}$;
%
%  Randomly pick $i_k \in \{1,2,\cdots, n\}$ and update weight: $${\theta}_k={\theta}_{k-1}-\alpha(g_{i_k}({\theta}_{k-1})-g_{i_k}(\hat{\theta})+\hat{\mu}).$$
%
% ${\hat{\theta}}_{s}={\theta}_m$ or ${\hat{\theta}}_{s}= {\theta}_k $ for randomly chosen $k \in \{1,2,\cdots, m-1\} $
%\end{algorithm}

\begin{algorithm}[t]
\caption{SVRG}
\textbf{Parameters:}s update frequency m and learning rate $\alpha$.\\
\textbf{Initialize:}$\tilde{\theta}$\\
%\begin{algorithmic}[1]
%\textbf{Output:} $X^*$. \\
\textbf{Iterate} for $s = 1,2,\cdots $
    $\tilde{\theta}={\tilde{\theta}_{s-1}}$\\
    $\tilde{\mu}=\frac{1}{n}\sum_{i=1}^{n} g_i $\\
    ${\theta}_{0}=\tilde{\theta} $\\
    \textbf{Iterate} Randomly pick $i_k \in \{1,2,\cdots, m\}$ and update weight: $${\theta}_k={\theta}_{k-1}-\alpha(g_{i_k}({\theta}_{k-1})-g_{i_k}(\tilde{\theta})+\tilde{\mu}).$$
  end\\
 \textbf{Option I:} set ${\tilde{\theta}}_{s-1}={\theta}_m$\\
 \textbf{Option II:}set  ${\tilde{\theta}}_{s-1}={\theta}_t$ for randomly chosen $t\in\{1,2,\cdots,m-1\}$

%\end{algorithmic}
\end{algorithm}
\subsection{SVRG++}
It is improved SVRG for non-strongly convex or sum-of-non-convex settings.%(https://arxiv.org/abs/1506.01972).
The objective function is $$F(x)=\sum _{i=1}^{n}f_{i}(x)+h(x),$$where $h(x)$ is the regularization term.
Let $g_i(x)={\nabla}_{x} f_i(x)$.

\begin{algorithm} %算法开始
\caption{SVRG++} %算法的题目
\label{SVRG++} %算法的标签
%\begin{algorithmic}[1] %此处的[1]控制一下算法中的每句前面都有标号
%\REQUIRE Text:${\tilde{x}}^0=x^{\phi},x_0^1=x^{\phi}$%输入条件(此处的REQUIRE默认关键字为Require,在上面已自定义为Input)
%% for loop
%\FOR{$s=1$ to $S$}
%\STATE ${\tilde{\mu}}_{s-1}=\sum _{i=1}^{n}g_i({\tilde{x}}^{s-1})$ \\
%       $m_s=2^sm_0$
%       \FOR {$t$ in $\{ 0,1, 2, \cdots, m_s -1\}$}
%       \STATE   Randomly pick $i_k \in \{1,2,\cdots, n\}$.\\
%        $$ \hat{w}=g_{i_k}({x}_{k}^s)-g_{i_k}(\hat{x}_{s-1})+\hat{\mu}_s$$,
%       $$x_{k+1}^s=argmin_{x}\{ h(x)+\frac{1}{2\gamma}\| x- x_k^s\|\}+<x,\hat{w}>\}$$
%\ENDFOR
%\FORALL{$i$ such that $0\leq i\leq 10$}
%\STATE carry out some processing
%\ENDFOR
%% while-loop
%\WHILE{some condition holds}
%\STATE carry out some processing
%\ENDWHILE
%% repeat-until loop
%\REPEAT
%\STATE carry out some processing
%\UNTIL{some condition is met}
%% infinite loop
%\LOOP
%\STATE this processing will be repeated forever
%\ENDLOOP
%\end{algorithmic}
1. During the $s$th epoch, $\hat{\mu}_{s-1}=\frac{1}{m_{s-1}}\sum_{i=1}^{m} g_i ({\hat{x}}_{s-1})$;\\
2. $m_s=2^s m_{0}$\\
3. For $k\in \{ 0,1, 2, \cdots, m_s -1\}$\\

 Randomly pick $i_k \in \{1,2,\cdots, n\}$ and update weight:
 $$ \hat{w}=g_{i_k}({x}_{k}^s)-g_{i_k}(\hat{x}_{s-1})+\hat{\mu}_s,$$
 $$x_{k+1}^s=argmin_{x}\{ h(x)+\frac{1}{2\gamma}\| x- x_k^s\|+<x,\hat{w}>\}$$\\

4. ${\hat x}_{s}=\frac{1}{m_s}\sum _{k=1}^{m_s}x_{k}^s$ ;\\
5. $x_0 ^{s+1}=x_{m_s}^s$
\end{algorithm}


\subsection{SCSD}
%[Less than a Single Pass]%(https://arxiv.org/abs/1609.03261).

As a member of the SVRG family of algorithms,
SCSG makes use of gradient estimates at two scales,
with the number of updates at the faster scale being governed by a geometric random variable.
Unlike most existing algorithms in this family, both the computation cost
and the communication cost of SCSG do not necessarily scale linearly with the
sample size $n$; indeed, these costs are independent of $n$ when the target accuracy is low.
\begin{algorithm} %算法开始
\caption{SCSD} %算法的题目
\label{SCSD} %算法的标签
1.During the $s$th epoch and $s\in \{ 1, 2, \cdots, T\}$, $\hat{\theta}={\hat{\theta}}_{s-1}$, $\hat{\mu}=\frac{1}{n}\sum_{i=1}^{m} g_i (\hat{\theta})$,${\theta}_0=\hat{\theta}$;
2.For $k\in \{ 1, 2, \cdots, M\}$, where $M \sim Geom(\frac{B}{B+1})$
  Randomly pick $i_k \in \{1,2,\cdots, n\}$ and update weight:
 $${\theta}_k={\theta}_{k-1}-\alpha(g_{i_k}({\theta}_{k-1})-g_{i_k}(\hat{\theta})+\hat{\mu}).$$

3.In this epoch:${\hat{\theta}}_{s}={\theta}_M$

4.(Strongly convex case)${\hat{\theta}}={\theta}_T$ or (Not strongly convex case)${\hat{\theta}}= \frac{1}{T}\sum _{k=1}^{T}{\theta}_k $
\end{algorithm}

\section{The Stochastic Properties}
%Neelakantan A, Vilnis L, Le Q V, et al. Adding Gradient Noise Improves Learning for Very Deep Networks[J]. arXiv: Machine Learning, 2015.
The randomness occurs in the stochastic gradient-based optimization, including:
\begin{itemize}
         \item Initial values: It is important to choose the initial values in practical aplications such as deep learning.
         \item The estimated gradient: The estimation is not based on reliable of statistical method due to the lack of the distribution of the  original full gradient.
         \item The noise of input data: The raw data is always filled with noise.
\end{itemize}


\end{document}
